	%Standard opsætning
\documentclass[a4paper,12pt]{article}
\usepackage[utf8]{inputenc}
\usepackage[T1]{fontenc}
\usepackage{amsfonts}
\usepackage{graphicx}
\usepackage{float}
\usepackage{hyperref}
\usepackage{listings,xcolor}
\usepackage[vlined, ruled, linesnumbered]{algorithm2e}
\usepackage{pdfpages}
%Til at indsætte pdf'er

\definecolor{dkgreen}{rgb}{0,.6,0}
\definecolor{dkblue}{rgb}{0,0,.6}
\definecolor{dkyellow}{cmyk}{0,0,.8,.3}

\hypersetup{
    colorlinks,
    citecolor=cyan,
    filecolor=cyan,
    linkcolor=cyan,
    urlcolor=cyan
}
\usepackage{amssymb}
%Danske symboler
\usepackage[danish]{babel}
%Matematik-ting
\usepackage{mathtools}
\usepackage{setspace}
%Halvanden linjeafstand
\onehalfspacing
%Sidehoved
\usepackage{fancyhdr}
\pagestyle{fancy}
%Rydder fancyheads(sidehoved) underlige tekst
\fancyhf{}
\setlength{\headheight}{15pt}
%Sidetal
\cfoot[]{\thepage} 
%Centreret sidehoved
\chead{Emil Lagoni, Erik Allin, Peter Spliid}

\renewcommand{\thesubsection}{\alph{subsection}}

\title{Objektorienteret Programmering og Design (OOPD) - \\ Uge 1}
\author{Emil Lagoni \\ Erik Allin \\ Peter Spliid}
\date{23. November 2013}

\begin{document}
\maketitle %insert the defined title
\thispagestyle{empty}
\setcounter{page}{0}
\newpage
% Nedenstående 2 linjer bruges til indholdsfortegnelse.
%\tableofcontents
%\newpage

% let's begin


\section*{P3.12}
Formålet med klassen er, at fjerne de kommaer, som forekommer ved input.\\
Et eksempel er at, der indtastes tallet 50,000, og klassens mål er så at printe tallet 50000 ud.

\section*{P3.13}
Klassen vil indsætte et komma, hvis tallet passer med det givne interval.\\
Hvis input er større eller mindre vil den skrive en passende besked.\\
Et eksempel er at, der indtastes tallet 99000, og klassens mål er så at printe tallet 99,000 ud.

\end{document}
