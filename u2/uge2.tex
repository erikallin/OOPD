	%Standard opsætning
\documentclass[a4paper,12pt]{article}
\usepackage[utf8]{inputenc}
\usepackage[T1]{fontenc}
\usepackage{amsfonts}
\usepackage{graphicx}
\usepackage{float}
\usepackage{hyperref}
\usepackage{listings,xcolor}
\usepackage[vlined, ruled, linesnumbered]{algorithm2e}
\usepackage{pdfpages}
%Til at indsætte pdf'er

\definecolor{dkgreen}{rgb}{0,.6,0}
\definecolor{dkblue}{rgb}{0,0,.6}
\definecolor{dkyellow}{cmyk}{0,0,.8,.3}

\hypersetup{
    colorlinks,
    citecolor=cyan,
    filecolor=cyan,
    linkcolor=cyan,
    urlcolor=cyan
}
\usepackage{amssymb}
%Danske symboler
\usepackage[danish]{babel}
%Matematik-ting
\usepackage{mathtools}
\usepackage{setspace}
%Halvanden linjeafstand
\onehalfspacing
%Sidehoved
\usepackage{fancyhdr}
\pagestyle{fancy}
%Rydder fancyheads(sidehoved) underlige tekst
\fancyhf{}
\setlength{\headheight}{15pt}
%Sidetal
\cfoot[]{\thepage}
%Centreret sidehoved
\chead{Emil Lagoni, Erik Allin, Peter Spliid}

\renewcommand{\thesubsection}{\alph{subsection}}

\title{Objektorienteret Programmering og Design (OOPD) - \\ Uge 2}
\author{Emil Lagoni \\ Erik Allin \\ Peter Spliid}
\date{1. December 2013}

\begin{document}
\maketitle %insert the defined title
\thispagestyle{empty}
\setcounter{page}{0}
\newpage
% Nedenstående 2 linjer bruges til indholdsfortegnelse.
%\tableofcontents
%\newpage

% let's begin

\section*{Kode}
\subsection*{makeBold}
Denne funktion bruges til at gøre velkomstteksten bold.


\subsection*{newBoard}
Ved at overskrive det bræt, som allerede findes, fås et tomt bræt.


\subsection*{printBoard}
Funktionen løber hele multiarrayet igennem og printer et mellemrum hvis feltet 
ikke er optaget, ellers indholdet i feltet.\\
Hvis det er det yderste felt til højre vil der tilføjes endnu en "|" for at få
et pænt bræt.


\subsection*{gameOver}
Ser om der er tre ens på stribe. (Desværre brute-forced)


\subsection*{play}
Starter spillet. Det kører indtil !gameOver() returnerer sand eller tallet fullBoard (der repræsenterer antallet af mulige pladser i spillet) counter ned til 0. Derefter spørger den om man vil spille igen.

\subsection*{isOccupied}
Ser om det ønskede felt er optaget, dvs. om der ligger et X eller O, da det er de gyldige brikker.

\subsection*{placeMove}
Først tjekkes der om feltet er optaget og hvis det er, så skriver i rødt at feltet er taget, ellers placerer den brikken.


\section*{Overvejelser}
\subsection*{Input-typer}
Vi har valgt at sige at bruger kun kan benytte sig af 0, 1, 2, X og O. Vi har valgt ikke
at teste for om brugers input er legalt og derfor kan programmet slutte uventet.

\subsection*{Output-format}
Da hele gruppen bruger en form for terminal som er ANSI-kompatibel, har vi
valgt af benytte os af dette. Der hvor det benyttes er ved ugyldige input og ved
starten af et nyt spil.

\end{document}
