\documentclass[a4paper,11pt]{article}
\usepackage[utf8]{inputenc}
\usepackage[T1]{fontenc}
%Sidehoved
\usepackage{fancyhdr}
\pagestyle{fancy}
%Rydder fancyheads(sidehoved) underlige tekst
\fancyhf{}
\setlength{\headheight}{15pt}
%Sidetal
\cfoot[]{\thepage}
%Centreret sidehoved
\chead{Emil Lagoni, Erik Allin, Peter Spliid}
%Halvanden linjeafstand
\usepackage{setspace}
\onehalfspacing


\title{\textbf{OOPD - u3}}
\author{Emil Lagoni\\Erik Allin\\Peter Spliid}

\begin{document}
\maketitle
\section*{Kode}
\subsection*{ComputerPlayer}
\subsubsection*{ComputerPlayer}
Opretter computerspilleren samt dens sværhedsgrad, som er en tilfældig integer mellem 0 (let) og svær (1).

\subsubsection*{makeMove}
Styrer om computeren laver et smart eller et dumt valg i sit træk. \\
Hvis sværhedsgraden i computerPlayer er = 1 foretages der kloge smarte valg hele spillet igennem, hvis 0 så dumme valg.


\subsubsection*{smartMove}
Definerer hvordan et smart valg tages. \\
Hvis tallet er et vilkårligt tal forskelligt fra en toer-potens - 1, så trækkes der en værdi, der får det tilbageværende antal kugler til at være lig en toer-potens - 1 (3, 7, 15, 31, 63). \\
Hvis antallet af kugler allerede er en toer-potens - 1, så tages der et tilfældigt lovligt tal. \\[5px]
\textit{avoidZero} er med til at tage værdien, hvis det sidste var tilfældet. Dette gøres da computeren ellers ville trække 0 kugler, hvilket er et ulovligt træk. \\[5px]
\textit{WONT\_GET\_THERE} er dels defineret fordi smartMove skal returnere noget, dels for at tjekke om der sker en fejl, der så kaster -1, som så vil være en ugyldig værdi.


\subsubsection*{dumbMove}
Definerer hvordan et dumt valg tages. \\
Her tages der blot et tilfældigt, (random) lovligt antal kugler fra bunken.

\subsection*{Game}
\subsubsection*{Game}
Starter spillet med at generere en integer mellem 10 og 100, der angiver størrelsen på heapet.

\subsubsection*{printHeap}
Printer heapets størrelse

\subsubsection*{getHeap}
Gør at heapets størrelse kan blive kaldt/brugt.

\subsubsection*{isLegit}
Tjekker om der bliver taget et lovligt antal kugler fra bunken.


\subsubsection*{remove}
Fjerner det antal kugler fra bunken, som der blev tjekket i isLegit.
Hvis tallet ikke er legalt bedes der om et nyt tal i det lovlige interval.


\subsection*{Nim}
Main metoden. \\
Denne afvikler spillet ved at sørge for at spillet bliver startet, computerspilleren bliver oprettet, der tjekkes for om nogen vinder, og spillet stopper i disse tilfælde.
Derudover der scannes for det indtastede nummer, samt at der printes de nødvendige beskeder for at en bruger kan finde ud af spillet.

\section*{Overvejelser}
Der er undervejs i opgaven blevet gjort flere overvejelser ud af hvordan vi får computeren til at holde styr på heapets størrelse. \\
Derudover har en implementation af sværhedsgrad el. dumbMove/smartMove også været noget der har givet mulighed for problemer. \\
Derudover var smartMoves valg af antal kugler der skulle trækkes også en udfordring.

\section*{Afprøvning}
Der er blevet lavet en række forsøg i forbindelse med at fejlteste programmet. \\
Først og fremmest bliver det testet om computeren både kan være smart og dum i samme spil. \\
Dette ses hurtigt ved at computeren i nogle tilfælde altid går efter at der er de bedste tal (2 i x'te - 1) tilbage i bunken. \\
Derudover har computeren undervejs været 'forsynet' med et system.out kald, der printede om den var dum eller smart. 
\\
I forbindelse med dette har vi også testet hvad den smarte computer gør i det tilfælde af, at vi foretager det samme valg, her tages et tilfældigt tal, som også er hensigten.
\\
Herudover er der kørt et stort antal spil (100+) for at være sikre på, at der ikke forekommer fejl hvis mængden i heapet bliver minut eller 0. \\
Dette har gjort, at vi i Nim-klassen har et tilfælde med (a.getHeap() == 1) og (a.getHeap() == 0), der henholdsvis tjekker om spilleren eller computeren har vundet.




\end{document}
