\documentclass[a4paper,11pt]{article}
\usepackage[utf8]{inputenc}
\usepackage[T1]{fontenc}
%Sidehoved
\usepackage{fancyhdr}
\pagestyle{fancy}
%Rydder fancyheads(sidehoved) underlige tekst
\fancyhf{}
\setlength{\headheight}{15pt}
%Sidetal
\cfoot[]{\thepage}
%Centreret sidehoved
\chead{Emil Lagoni, Erik Allin, Peter Spliid}
%Halvanden linjeafstand
\usepackage{setspace}
\onehalfspacing


\title{\textbf{OOPD - u4}}
\author{Emil Lagoni\\Erik Allin\\Peter Spliid}

\begin{document}
\maketitle
\section*{Kode}
\subsection*{Appointment}
\subsubsection*{Appointment}
Definerer en appointment ved at den har et årtal, en måned, et dag og en beskrivelse.


\subsubsection*{getEvent}
Metoden skal kunne returnere appointments fra en specifik dato.


\subsubsection*{occursOn}
Tjekker om der befinder sig en begivenhed på den pågældende dag.



\subsection*{OneTime}
\subsubsection*{Onetime}
Opretter en oneTime begivenhed.

\subsubsection*{print}
Printer alle oneTime begivenheder ud.

\subsubsection*{occursOn}
Tjekker om alle oneTime appointments foregår på en bestemt dato.


\subsection*{Daily}
\subsubsection*{Daily}
Opretter en daily begivenhed.

\subsubsection*{print}
Printer alle daily begivenheder ud.

\subsubsection*{occursOn}
Tjekker om alle daily appointments foregår på en bestemt dato.


\subsection*{Monthly}
\subsubsection*{Monthly}
Opretter en monthly begivenhed.

\subsubsection*{print}
Printer alle monthly begivenheder ud.

\subsubsection*{occursOn}
Tjekker om alle monthly appointments foregår på en bestemt dato.


\subsection*{Calendar}
Holder styr på hvilken fil der skal skrives og hentes data fra (KalenderFil.csv). Derudover sørger klassen for at filen ikke bliver overskrevet, men at der blot skrives mere ind. \\
Klassen beder også brugeren om at indtaste dag, måned, år og beskrivelse på begivenheden der indtastes samt om det er en O (onetime), D (daily) eller M (monthly) begivenhed. \\
Til sidst skrives der ud hvad filen indeholder, og programmet terminerer.

\section*{Overvejelser}
Der har undervejs været en lang række overvejelser. \\
Den nok største har været: "\textit{Hvad skal de forskellige klasser gøre, og hvordan skal de i forbindelse med disse funktioner arbejde sammen?}" \\
Herudover har vi undervejs overvejet hvordan kalenderen skulle repræsenteres (indskrives/hentes ud) i filen. Dette gjorde også at vi gik fra et .txt dokument til et .csv dokument, da en kalender burde kunne gøres pæn i et excel miljø.




\end{document}