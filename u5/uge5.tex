\documentclass[a4paper,11pt]{article}
\usepackage[utf8]{inputenc}
\usepackage[T1]{fontenc}
%Sidehoved
\usepackage{fancyhdr}
\pagestyle{fancy}
%Rydder fancyheads(sidehoved) underlige tekst
\fancyhf{}
\setlength{\headheight}{15pt}
%Sidetal
\cfoot[]{\thepage}
%Centreret sidehoved
\chead{Emil Lagoni, Erik Allin, Peter Spliid}
%Halvanden linjeafstand
\usepackage{setspace}
\onehalfspacing


\title{\textbf{OOPD - u5}}
\author{Emil Lagoni\\Erik Allin\\Peter Spliid}

\begin{document}
\maketitle

\section*{Kode}
Se javadoc

\section*{Overvejelser}
\subsection*{getLinks}
Først har vi valgt at søge for links som har tekst. \\
Da udviklere har forskellig kodestil, har vi valgt at der søges i "case insensitve".

\section*{Testning}
For at teste programmet er der blevet kørt de forskellige funktioner på bl.a. Google.dk og Youtube.com \\
Ved funktionen der skriver en URL's data til en fil ("s" i CLI'en) lykkes det at smide alt koden korrekt ind, 
der er altså ikke noget der står forkert i forhold til det originale, ikke forkerte newlines mm. \\
Ved print-funktionen ("p i CLI'en), der printer indholdet af den seneste lavede fil ind i terminalen 
foregår tingene også korrekt. \\
Endeligt har der været meget arbejde med printningen af hyperlinks ("l" i CLI'en), der tager et filnavn på en
tidligere lavet fil, og printer alle hyperlinks, samt deres "detaljer/tekst/navn" ind i terminalen. \\
Dette virker efter hensigten, og der laves nye linjer på de korrekte tidspunker. \\
Endeligt er der blevet testet at programmet kan lukke ("q" i CLI'en), der ganske enkelt fungerer ved et "break". \\
Denne funktion virker som den skal.


\end{document}
